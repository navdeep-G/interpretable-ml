\section{Surrogate Models}

\begin{frame}[c]
\Huge{\centerline{Surrogate Models}}
\end{frame}

%----------------------------------------------------------------------------------------

\begin{frame}\frametitle{Surrogate Models}
	\begin{itemize}
	        \item A \textit{surrogate model} is a data mining and engineering technique in which a simple model is used to explain another complex model.
		\item Given our learned function $g$ and set of predictions, $g(\mathbf{X}) = \hat{\mathbf{Y}}$, we can train a surrogate model $h$:
			
			\begin{equation}
			\begin{aligned}
				 \mathbf{X},\hat{\mathbf{Y}} \xrightarrow{\mathcal{A}_{\text{surrogate}}} h\
			\end{aligned}
			\end{equation}
		\item Ideally $h(\mathbf{X}) \approx g(\mathbf{X})$, however there exist few guarantees that $h(\mathbf{X})$ accurately represents $g$.
		\item To preserve interpretability, the hypothesis set for $h$ is often restricted to linear models or decision trees.
	\end{itemize}
\end{frame}

%----------------------------------------------------------------------------------------

\begin{frame}\frametitle{Surrogate Models (Cont.)}
	\begin{itemize}
	        \item While any model can act as a surrogate model, surrogate models are chosen because they are easy for a human to interpret and explain. Surrogate models enhance transparency by providing:
	        \bigskip
	        \begin{itemize}
	        \item Specific insights into the mechanism and results of a complex model.
	        \item Global or local interpretations of a complex model.
	        \item Visualizations that are easy to understand and compare.
	        \end{itemize}
	        \bigskip
	        \item Surrogate models are limited to linear models, decision trees, and random forests.
	
	\end{itemize}
\end{frame}

%----------------------------------------------------------------------------------------

